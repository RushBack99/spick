%----------------------------------------------------------------------------------------------
%-------------------------------     1. Differentialrechnung    -------------------------------
%----------------------------------------------------------------------------------------------
\section{Differentialrechnung}{444}

\begin{tabular}{ll | ll} 
    Kurvenuntersuchungen & S. 261 & Taylorreihe & S. 455 
\end{tabular}

%----------------------------------------------------------------------------------------------
%-------------------------------     1.1 Differentialquotient    ------------------------------
%----------------------------------------------------------------------------------------------
\subsection{Differenzialquotient}{444}

Die Ableitung $f'(x)$ der Funktion $f(x)$ im Punkt $x_0$ entspricht der Steigung der Tangente an $f(x)$ im Punkt $x_0$.

\begin{tabular}{lll}
Differenzenquotient: 
& $ \displaystyle \frac{f(x + h) - f(x)}{h} = \frac{f(x + \Delta x) - f(x)}{\Delta x} $ & \\[1em]

\textbf{Differenzialquotient:} 
& $ \displaystyle f'(x) = \lim\limits_{\Delta x \to 0} \frac{f(x + \Delta x) - f(x)}{\Delta x} = \frac{dy}{dx} = \tan \alpha$
\end{tabular}\\

Die Existenz der Ableitung einer Funktion $f(x)$ für die Werte der Variable $x$ ist gegeben, wenn diese Werte  der Differenzialquotient einen endlichen Wert besitzt.

%----------------------------------------------------------------------------------------------
%----------------------     1.2 Tangente / Normale / Zwischenwinkel    ------------------------
%----------------------------------------------------------------------------------------------
\subsection{Tangente / Normale / Zwischenwinkel}

\vspace{-0.2cm}
\renewcommand{\arraystretch}{2}
\begin{tabular}{ll}
    Tangente:       & $y = f'(x_0) \cdot (x - x_0) + y_0$ \\
    Normale:        & $y = -\frac{1}{f'(x_0)} \cdot (x - x_0) + y_0$ \\	
    Zwischenwinkel: &  $\tan(\alpha) = \frac{m_1 - m_2}{1 + m_1 \cdot m_2}$ $\Rightarrow$ Winkel gegen Uhrz.\\
                    &  $\tan(\alpha) =  \frac{m_2 - m_1}{1 + m_1 \cdot m_2}$ $\Rightarrow$ Winkel im Uhrzeigersinn
\end{tabular}
\renewcommand{\arraystretch}{1}

%----------------------------------------------------------------------------------------------
%----------------------------     1.3 Einseitige Ableitungen    -------------------------------
%----------------------------------------------------------------------------------------------
\subsection{Einseitige Ableitungen}{445}

\vspace{-0.2cm}
\begin{minipage}[c]{0.48\columnwidth}
    $$ \text{Rechtsseitig:} \quad  f'_r(x_0) = \lim\limits_{x \to x_0^+} f'(x) $$
\end{minipage}
\hfill
\begin{minipage}[c]{0.48\columnwidth}
    $$ \text{Linksseitig:} \quad f'_l(x_0) = \lim\limits_{x \to x_0^-} f'(x) $$
\end{minipage}

\renewcommand{\arraystretch}{1.4}
\begin{tabular}{ll}
    $f'_r(x_0) = f'_l(x_0)$     & $\Rightarrow$ Konvergenz \\
    Alle anderen Fälle          & $\Rightarrow$ unbestimmte Divergenz $\Rightarrow$ keine Ableitung!
\end{tabular}
\renewcommand{\arraystretch}{1}

\begin{center}
    \includegraphics[width=1\columnwidth]{images/konvergenz-divergenz.PNG}
\end{center}

%----------------------------------------------------------------------------------------------
%-------------------------------     1.4 Ableitungsregeln    ----------------------------------
%----------------------------------------------------------------------------------------------
\subsection{Ableitungsregeln}{445-448}

\subsubsection{Elementare Regeln}

\vspace{-0.2cm}
\renewcommand{\arraystretch}{1.6}
\begin{tabular}{lll}
    Konstanten:     & $c =$ konst                           & $c' = 0$ \\
    Faktor:         & $f(x) = c \cdot x^2$                  & $f'(x) = c \cdot 2 \, x $ \\
    Summen:         & $(u(x) + v(x) - w(x))' $              & $u'(x) + v'(x) - w'(x)$ \\
    Potenzen:       & $f(x) = x^{\alpha}$                   & $f'(x) = \alpha \cdot x^{\alpha - 1}$ \\
                    & $f(x) = x^3$                          & $f'(x) = 3 \, x^2$       
\end{tabular}


\subsubsection{Produktregel}
\vspace{-0.3cm}
$$ (f(x) \cdot g(x))' = f'(x) \cdot g(x) + f(x) \cdot g'(x) $$ 


\subsubsection{Quotientenregel}
\vspace{-0.3cm}
$$ \left( \frac{u(x)}{v(x)} \right) ' = \frac{u'(x) \cdot v(x) - u(x) \cdot v'(x)}{v(x) ^2} \quad \text{$\Rightarrow$ als Produkt schreiben} $$
$$ u(x) \cdot \left( \frac{1}{v(x)} \right) ' =  u'(x) \cdot \frac{1}{v(x)} + u(x) \cdot \frac{- v'(x)}{v(x)^2} $$


\subsubsection{Kettenregel}
\vspace{-0.3cm}
$$ g(f(x))' =  f'(x) \cdot g'(x) $$ 

\subsubsection{Allgemeine Logarithmus-Ableitung}
\vspace{-0.3cm}
$$ (\log_b(x))' = \left( \frac{\ln(x)}{\ln(b)} \right)' = \frac{1}{\ln(b)} \cdot (\ln(x))' = \frac{1}{\ln(b)} \cdot \frac{1}{x} $$


\subsubsection{Umkehrfunktion}
\vspace{-0.2cm}
$$ (f^{-1}(y_0))' = \frac{1}{f'(x_0)} = \frac{1}{f'(f^{-1}(y_0))} $$ 


%----------------------------------------------------------------------------------------------
%-------------------------------     1.5 Ableitungsregeln    ----------------------------------
%----------------------------------------------------------------------------------------------
\subsection{Wichtige Ableitungen}{446}

\begin{tabular}{|l|l||l|l|}
    \hline
    Funktionen & Ableitung & Funktionen & Ableitung \\
    \hline
    $ C \;$(Konstante)                          & $ 0 $                                     & $\tan x \; \, (x \neq (2k+1)\frac{\pi}{2}, \; k \in \mathbb{Z})$  & $ \frac{1}{\cos^2 x}  $ \\
    $ x $                                       & $ 1 $                                     & $\cot x \; \, (x \neq k\pi, \; k \in \mathbb{Z})$                 & $ -\frac{1}{\sin^2 x} $ \\
    $ x^n $                                     & $ n \cdot x^{n-1} $                       & $\sec x$                                                          & $ \frac{\sin x}{\cos^2 x} $ \\
    $ \frac{1}{x} $                             & $ -\frac{1}{x^2} $                        & $\csc x$                                                          & $ -\frac{\cos x}{\sin^2 x} $ \\
    $ \frac{1}{x^n} $                           & $ -\frac{n}{x^n+1} $                      & $\sin^{-1} x \quad  (\abs{x} < 1)$                                & $ \frac{1}{\sqrt{1-x^2}} $ \\
    $ \sqrt{x} $                                & $ \frac{1}{2 \cdot \sqrt{x}} $            & $\cos^{-1} x \quad  (\abs{x} < 1)$                                & $ -\frac{1}{\sqrt{1-x^2}} $ \\
    $ \sqrt[n]{x} $                             & $ \frac{1}{n \cdot \sqrt[n]{x^{n-1}}} $   & $\tan^{-1} x$                                                     & $ \frac{1}{1+x^2} $ \\
    $ e^x $                                     & $ e^x$                                    & $\cot^{-1} (x)$                                                   & $ -\frac{1}{1+x^2} $ \\
    $ e^{bx} \quad (b \in \mathbb{R}) $         & $ b \cdot e^{bx} $                        & $\sec^{-1} x \quad  (x > 1)$                                      & $ \frac{1}{x\sqrt{x^2-1}} $ \\
    $ a^x \quad (a > 0)$                        & $ a^x \cdot \ln a $                       & $\csc^{-1} x \quad  (x > 1)$                                      & $ -\frac{1}{x\sqrt{x^2-1}} $ \\
    $ a^{bx} \quad (b \in \mathbb{R},\; a > 0)$ & $ ba^{bx} \cdot \ln a $                   & $\sinh x $                                                        & $ \cosh x $ \\
    $ \ln x \quad (x > 0)$                      & $ \frac{1}{x} $                           & $\cosh x $                                                        & $ \sinh x $ \\
    $ \sin x $                                  & $ \cos x $                                & $\tanh x$                                                         & $ \frac{1}{\cosh^2 x} $ \\
    $ \cos x $                                  & $ -\sin x $                               & $\coth x \quad  (x \neq 0)$                                       & $ -\frac{1}{\sinh^2 x} $ \\
    \hline
\end{tabular}

%----------------------------------------------------------------------------------------------
%-----------------------------     1.6 Approximationsfehler    --------------------------------
%----------------------------------------------------------------------------------------------
\subsection{Approximationsfehler}

Die Fehler beziehen sich auf den Arbeitspunkt (z.B. $x_0$)

\renewcommand{\arraystretch}{2}
    \begin{tabular}{lll}
    Absoluter Fehler:   & $R(x) = \Delta y - \diff y = f(x) - \hat f(x)$                             & Einheit von $y$ \\
    Relativer Fehler:   & $R(x) = \frac{\Delta y - \diff y}{y_0} = \frac{f(x) - \hat f(x)}{f(x_0)}$  & einheitenlos
\end{tabular}
\[
    \Delta y = f(x_0 + \Delta x)- f(x_0) \quad \quad dy = f'(x_0)\cdot \Delta x \quad \quad \hat f(x) = f(x_0)+ f'(x_0)(\Delta x)  
\]
\renewcommand{\arraystretch}{1.5}

\subsubsection*{Beispiel:  Approximationsfehler}
\[
    f(x) = \sqrt{x}\text{,} \quad\quad  \text{für} \;x_0 = 4 \;\text{mit}\; \Delta x = 0.1
\]
\begin{tabular}{llclcll}
    \\
    $ \Delta y$ &$=$& $f(4+0.1)-f(4)$ &$=$& $\sqrt{4.1} - \sqrt{4}$ &$=$& $0.024845$\\
    $ dy$ &$=$& $f'(4) \cdot 0.1$ &$=$& $\frac{1}{2\cdot \sqrt{4}} \cdot 0.1$ &$=$& $0.025 $ \\
    $ R_1(x)$ &$=$& $\abs{\Delta y - dy}$ &$=$& $\abs{0.024845 - 0.025}$ &$=$& $0.0000155$\\
    $ f(x)$ &$=$& $f(4+0.1)$ &$=$& $\sqrt{4.1}$ &$=$& $2.024845$\\
    $ \hat f(x)$ &$=$& $f(4)+ f'(4)(0.1)$ &$=$& $\sqrt{4} + \frac{1}{2\cdot \sqrt{4}} \cdot 0.1$ &$=$& $2.025 $ \\
    $ R_2(x)$ &$=$& $\abs{\Delta y - dy}$ &$=$& $\abs{2.024845 - 2.025}$ &$=$& $0.0000155$\\
    \hline
              &   & $ R_1(x)$ &$=$& $ R_2(x)$ &$=$& $\fbox{0.0000155}$
\end{tabular}

%----------------------------------------------------------------------------------------------
%------------------------------     1.7 Fehlerfortpflanzung    --------------------------------
%----------------------------------------------------------------------------------------------
\subsection{Fehlerfortpflanzung}{866}

\renewcommand{\arraystretch}{2}
\begin{tabular}{llll}
    Absolut: & $\cbl{\Delta x \rightarrow \Delta y}$                & & 
               $\cgn{\Delta y \approx dy = f'(x_0) \cdot \diff x}$                                              \\
    Relativ: & $\cor{\Delta x \rightarrow \frac{\Delta y}{y_0}}$    & &  
               $\cvt{\frac{\Delta y}{y_0} \approx \frac{\diff y}{y_0} = \frac{f'(x_0) \cdot \diff x}{f(x_0)}}$ 
\end{tabular}
\renewcommand{\arraystretch}{1}

\begin{minipage}[c]{0.65\columnwidth}

    \renewcommand{\arraystretch}{1.5}
    \begin{tabular}{|@{}| c | c | c |@{}}
    \hline
                                                            & $\Delta x = \diff x$ (abs)    & $\frac{\Delta x}{x} = \frac{\diff x}{x}$ (rel)    \\
    \hline
    $\Delta y \approx \diff y$ (abs)                        & \cbl{A}                       & \cgn{B}                                           \\
    \hline
    $\frac{\Delta y}{y} \approx \frac{\diff y}{y}$ (rel)    & \cor{C}                       & \cvt{D}                                           \\
    \hline
    \end{tabular}	
    \renewcommand{\arraystretch}{1}				
\end{minipage}
\hfill
\begin{minipage}[c]{0.34\columnwidth}
    \raggedright
    $\Rightarrow$ Tabelle ist bidirektional
    $\Rightarrow$ (Umkehrfunktionen)
\end{minipage}
