%----------------------------------------------------------------------------------------------
%----------------------------------     3 Folgen & Reihen    ----------------------------------
%----------------------------------------------------------------------------------------------

\section{Folgen und Reihen}{19, 470}

\begin{tabular}{ll | ll} 
    Exponentialgesetze:     & S. 8      & Logarithmengesetze:   & S. 9  \\
    Grenzwerts"atze:        & S. 471    &                               \\
\end{tabular}

%----------------------------------------------------------------------------------------------
%--------------------------     3.1 Spezielle Folgen und Reihen     ---------------------------
%----------------------------------------------------------------------------------------------
\subsection{Spezielle Folgen und Reihen}{20}

\renewcommand{\arraystretch}{1.4}
\begin{tabular}{llll}
    Arithmetische Folge:    & Bronstein S.19            & $a_{i+1} = a_i + i \cdot d$       & $d = \Delta a_i =  a_{i+1} - a_i$       \\
    Geometrische Folge:     & Bronstein S.20            & $a_{i} = q^i \cdot a_0 $          & $q = \frac{a_{i+1}}{a_i}$ \\
    Spezielle Reihen        & Bronstein S.20 
\end{tabular}

%----------------------------------------------------------------------------------------------
%--------------------------     3.2 Beschränktheit / Monotonie     ----------------------------
%----------------------------------------------------------------------------------------------
\renewcommand{\arraystretch}{1}	
\subsection{Beschränktheit / Monotonie}{51, 470}

\subsubsection{Beschränktheit}
\hspace{0.2cm}
$W_f \subset [a ; b]$ und $a, \, b \in \mathbb{R}$		


\subsubsection{Monotonie}
\hspace{0.2cm}
\begin{tabular}{cccc}
    \toprule
    $d^{(1)} \ge 0$ & $q^{(2)} \ge 1$ & monoton wachsend & $\uparrow$ \\
    \midrule
    $d > 0$ & $q > 1$ & streng monoton wachsend & $\Uparrow$ \\
    \midrule
    $d \le 0$ & $0 < q \le 1$ & monoton fallend & $\downarrow$ \\
    \midrule
    $d < 0$ & $0 < q < 1$ & streng monoton fallend & $\Downarrow$ \\
    \bottomrule
\end{tabular}\\

\begin{tabular}{ll}
    $(1)$ $d$ arithmetische Folge &
    $(2)$ $q$ geometrische Folge ($a_1 > 0$)
\end{tabular}

%----------------------------------------------------------------------------------------------
%--------------------------     3.3 Konvergenz / Divergenz     ----------------------------
%----------------------------------------------------------------------------------------------
\subsection{Konvergenz / Divergenz}{471}

\subsubsection{Konvergenz}

Es existiert ein Grenzwert g $\in \mathbb{R}$
$$ \text{Toleranzungleichung:} \quad \vert a_n - g \vert < \varepsilon  \text{ mit } \varepsilon > 0 $$
Gesucht ist ein $n_0$, ab welchem alle Werte von $n \geq n_0$ in $U_\varepsilon(g)$ liegen.


\subsubsection{Bestimmt divergent gegen +$\infty$}

$$ \text{Ungleichung:} \quad f_n > K \text{ wenn } n \geq  n_0 \text{ f"ur } K > 0 $$


\subsubsection{Bestimmt divergent gegen -$\infty$}

$$ \text{Ungleichung:} \quad f_n < k \text{ wenn } n \geq  n_0 \text{ f"ur } k < 0 $$


\subsubsection{Unbestimmt divergent}

$$ \text{Alles, was nicht konvergent oder bestimmt divergent ist} $$

%----------------------------------------------------------------------------------------------
%----------------------------------     3.4 Grenzwerte     ------------------------------------
%----------------------------------------------------------------------------------------------
\subsection{Grenzwerte}{471}

Die Folgen ${\textcolor{red}{a_n}}$ und ${\textcolor{blue}{b_n}}$ sind konvergent!\vspace{0.5em}
\\
\renewcommand{\arraystretch}{2}
\begin{tabular}{ll}
    $\lim\limits_{n \to \infty} \abs{\textcolor{red}{a_n}}$ \quad \;\;\; $=$   $\abs{\lim\limits_{n \to \infty} \textcolor{red}{a_n}} $\\

    $\lim\limits_{n \to \infty} \left(\textcolor{red}{a_n} + \textcolor{blue}{b_n}\right)$  $=$   $\lim\limits_{n \to \infty} \textcolor{red}{a_n} + \lim\limits_{n \to \infty} \textcolor{blue}{b_n}$ &
    $\lim\limits_{n \to \infty} \left(\textcolor{red}{a_n} - \textcolor{blue}{b_n}\right)$  $=$   $\lim\limits_{n \to \infty} \textcolor{red}{a_n} - \lim\limits_{n \to \infty} \textcolor{blue}{b_n}$\\
    
    $\lim\limits_{n \to \infty} \left(\textcolor{red}{a_n} \cdot \textcolor{blue}{b_n}\right)$ \; $=$   $\lim\limits_{n \to \infty} \textcolor{red}{a_n} \cdot \lim\limits_{n \to \infty} \textcolor{blue}{b_n}$ &
    $\lim\limits_{n \to \infty} \left(\frac{\textcolor{red}{a_n}}{\textcolor{blue}{b_n}}\right)$ \quad \;\:\, $=$   $\frac{\lim\limits_{n \to \infty} \textcolor{red}{a_n}}{\lim\limits_{n \to \infty} \textcolor{blue}{b_n}}$, \quad für $\lim\limits_{n \to \infty} \textcolor{blue}{b_n} \neq 0$ 
\end{tabular}

\newpage

%----------------------------------------------------------------------------------------------
%----------------------------------     3.4 Grenzwerte     ------------------------------------
%----------------------------------------------------------------------------------------------
\subsection{Rechnen \texorpdfstring{$\Longrightarrow \infty$}{→ ∞}}

\subsubsection{Bestimmte Formen}

\renewcommand{\arraystretch}{2}
\begin{tabular}{lll}
    $\infty + \infty = \infty$      & $-\infty - \infty = -\infty$          & $0 \cdot [a,b] = 0 \cdot \text{beschr"ankt} = 0$  \\
    $g + \infty = \infty$           & $g - \infty = -\infty$                & 	$(g \in \mathbb{R})$                            \\
    $\infty \cdot \infty = \infty$  &	$-\infty \cdot (\infty) = -\infty$  &
    $g \cdot \infty =
    \begin{cases}
        \infty   & g > 0 \\ 
        -\infty & g < 0 
    \end{cases}$        \\ 		
    $\frac{1}{\infty} = 0$          &	$\frac{g}{\infty} = 0$              & 	$g \in \mathbb{R}$	                            \\
    $\frac{\infty}{0+} = \infty$	& $\frac{\infty}{0-} = -\infty$
    & $\frac{\infty}{g} =
    \begin{cases}			
        \infty & g > 0 \\
        -\infty & g < 0
    \end{cases} $ \\
    $\frac{1}{0+} = \infty$         & $\frac{1}{0-} = -\infty$	            & $g \in \mathbb{R} - {0}$                          \\
\end{tabular}

\vspace{0.2cm}

\begin{tabular}{lcl}
    $\frac{g}{0+} = 
    \begin{cases}					
        \infty & g > 0 \\
        -\infty & g < 0
    \end{cases} $ & 
    &
    $\frac{g}{0-} =
    \begin{cases}				
        -\infty & g > 0 \\
         \infty & g < 0
    \end{cases}$ \\ 
\end{tabular}
\renewcommand{\arraystretch}{1}

\subsubsection{Unbestimmte Formen}


\begin{tabular}{lllllll}
    $\frac{0}{0} = ?$       & $\frac{\infty}{\infty} = ?$   & $\infty \cdot 0 = ?$  & $\infty^0 = ?$ & $0 \cdot \infty = ?$    & $\infty - \infty = ?$         & $0^0 = ?$     \\
\end{tabular}

\vspace{0.2cm}

\begin{tabular}{ll}
    $1^{\infty} = ?$    & Ausser 1 ist eine Konstante, dann gilt $1^{\infty} = 1 $
\end{tabular}

%----------------------------------------------------------------------------------------------
%--------------------------     3.6 Grenzwerte gegen unendlich     ----------------------------
%----------------------------------------------------------------------------------------------
\subsection{Grenzwerte \texorpdfstring{$\longrightarrow \infty$}{→ ∞}}
\[
    \lim\limits_{n \to \infty}\left(\frac{-2n^2+4n-5}{8n^2-3n+7}\right)
\]
\begin{tabular}{lcl}
    $ \displaystyle{f(n)= \frac{-2n^2+4n-5}{8n^2-3n+7}}$                    & Algebraisch erweitern mit $\frac{1}{n^2}$: & $\displaystyle{f(n)=\frac{-2+\frac{4}{n}-\frac{5}{n^2}}{8-\frac{3}{n}+\frac{7}{n^2}}}$\\
    $ \displaystyle{f(n)=\frac{-2+\frac{4}{n}-\frac{5}{n^2}}{8-\frac{3}{n}+\frac{7}{n^2}}}$ & $\xrightarrow{n \to \infty}$               & $\displaystyle{f(n)=\frac{-2}{8} = -\frac{1}{4}}$
\end{tabular}\\

\[
    \lim\limits_{n \to \infty}\left(\frac{-2n^2(4n-5)}{8n^2(-3n+7)}\right)
\]
\begin{tabular}{lcl}
    $ \displaystyle{f(n)= \frac{-2n^2(4n-5)}{8n^2(-3n+7)}}$                 & Algebraisch erweitern mit $\frac{1}{n^3}$: & $\displaystyle{f(n)=\frac{-8+\frac{10}{n}}{-24+\frac{56}{n}}}$\\
    $ \displaystyle{f(n)=\frac{-8+\frac{10}{n}}{-24+\frac{56}{n}}}$         & $\xrightarrow{n \to \infty}$               & $\displaystyle{f(n)=\frac{-8}{-24} = \frac{1}{3}}$ 
\end{tabular}

%----------------------------------------------------------------------------------------------
%--------------------------------     3.7 Bolzano-Prinzip     ---------------------------------
%----------------------------------------------------------------------------------------------
\subsection{Bolzano-Prinzip}

\textbf{Jede beschränkte, monotone Zahlenfolge ist konvergent!}

\example{Grenzwert von rekursiver Folge}

$$ \text{Folge:} \quad a_1 = \frac{1}{4} ; a_{n+1} = a_n^2 + \frac{1}{4} $$ 

%TODO: enumerate Layout
\begin{enumerate}
	\item Monotonie \\
		Beweisen mit Ansatz $a_{n+1} \geq a_n$ bzw. $a_{n+1} \leq a_n \rightarrow$ Induktion

	\item Beschränktheit \\
		Erste Schranke = Erster Wert der Reihe \\
		Zweite Schranke: Annahme, es gibt Grenzwert $g$ und er ist Supremum bzw. Infimum

	\item Beweisen (oder widerlegen), dass $g$ $\sup$ bzw. $\inf$ ist \\
		Ansatz: $a_n \leq g$ bzw. $a_n \geq g$ mit vollständiger Induktion beweisen
\end{enumerate}

\columnbreak

